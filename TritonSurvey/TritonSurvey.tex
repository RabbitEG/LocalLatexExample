\documentclass[UTF8,12pt]{ctexart}

\usepackage{amsmath,amssymb}
\usepackage{geometry}
\ifdefined\TRITONDRAFT
  \PassOptionsToPackage{draft}{graphicx}
\fi
\usepackage{graphicx}
\usepackage{float}
\usepackage{listings}
\usepackage{xcolor}
\usepackage[hidelinks]{hyperref}
\usepackage{placeins}
\usepackage{subcaption}
\usepackage{booktabs}
\usepackage{tabularx}

\geometry{a4paper,margin=2.2cm}
\setlength{\parindent}{2em}
\setlength{\parskip}{0.4em}

% Keep section titles left-aligned.
\ctexset{
  section = {format=\Large\bfseries\raggedright},
  subsection = {format=\large\bfseries\raggedright},
  subsubsection = {format=\normalsize\bfseries\raggedright},
  appendix = {name=附录}
}

\definecolor{codegray}{rgb}{0.45,0.45,0.45}
\definecolor{codeblue}{rgb}{0.0,0.25,0.75}
\definecolor{codegreen}{rgb}{0.0,0.5,0.0}

\lstset{
  basicstyle=\ttfamily\small\linespread{0.9}\selectfont,
  breaklines=true,
  frame=single,
  numbers=left,
  numberstyle=\tiny\color{codegray},
  keywordstyle=\color{codeblue},
  commentstyle=\color{codegreen},
  showstringspaces=false
}

\graphicspath{{}{figures/}}

\begin{document}
\sloppy

\begin{titlepage}
\begin{center}
\linespread{1.2}\huge {\bfseries Fudan University}\\[1cm]
\linespread{1}
\IfFileExists{figures/fudan-name.pdf}{%
  \includegraphics[width=10cm]{figures/fudan-name.pdf}\\[3cm]%
}{%
  \vspace{3.5cm}%
}

\linespread{1.8}\Large {\bfseries Parallel Computing}\\
\linespread{1.8}\Large {\bfseries Language Survey}\\[0.6cm]
\linespread{1.6}\huge {\bfseries TritonSurvey}\\[3.0cm]

\Large 姓名:(示例作者)\\
\Large 学号:(示例学号)\\[2.0cm]

\normalsize \the\year 年 \the\month 月
\end{center}
\end{titlepage}

\clearpage
\renewcommand{\abstractname}{\large 摘要}
\begin{abstract}
\thispagestyle{empty}
本文档是一个可直接编译的中文示例稿,展示如何用 \texttt{latexmk + Makefile} 管理文档构建。内容围绕 Triton(面向 GPU 的 Python DSL)做一个“语言调研”风格的简要介绍,并提供少量示例引用与参考文献。
\end{abstract}

\newpage
\pagestyle{empty}
\tableofcontents
\thispagestyle{empty}
\newpage
\pagestyle{plain}
\pagenumbering{arabic}
\setcounter{page}{1}

\section{背景与目标}
在 GPU 编程生态里,开发者常见的选择包括 CUDA C++、OpenCL,以及更高层的编程抽象(如基于 Python 的 DSL)。
Triton 的定位可以粗略理解为:在保持性能可控的同时,提供更接近 Python 科研/工程栈的开发体验 \cite{triton}.

本示例文档的目标是:
\begin{itemize}
  \item 提供一个最小但完整的可编译结构(目录、公式、表格、代码块、图片)。
  \item 演示通过 \texttt{latexmk} 自动决定“跑几遍”,并把产物集中到 \texttt{build/}。
\end{itemize}

\subsection{一个极简的性能直觉}
下面给出一个非常粗糙的估算式(仅用于示例排版):
\begin{equation}
  T \approx \frac{\text{Bytes Moved}}{\text{Memory Bandwidth}} + \frac{\text{FLOPs}}{\text{Compute Throughput}}.
\end{equation}
在工程实践中,通常会结合硬件文档和编程指南来验证这些直觉 \cite{cuda_guide}.

\section{Triton 的核心概念(示例)}
\subsection{编程模型}
Triton 常用的描述方式是“以 block/tile 为单位的张量计算”:将大问题拆成若干 tile,每个 program 实例负责处理一块 tile,
并通过显式的内存访问模式来争取更好的带宽利用率。

\subsection{示例图片}
图 \ref{fig:placeholder} 只是为了演示图像插入与构建流程。
在 \texttt{docs/} 子工程里,我们把图片文件放在 \texttt{docs/figures/} 下,避免与仓库其它目录冲突。

\begin{figure}[H]
  \centering
  \IfFileExists{figures/placeholder.pdf}{%
    \includegraphics[width=0.72\textwidth]{figures/placeholder.pdf}%
  }{%
    \fbox{\parbox[c][4cm][c]{0.72\textwidth}{\centering Placeholder Figure}}%
  }
  \caption{示例图片(可用 placeholder.pdf,也可用占位框)}
  \label{fig:placeholder}
\end{figure}

\subsection{示例表格}
\begin{table}[H]
  \centering
  \begin{tabular}{lcc}
    \hline
    方案 & 开发体验(主观) & 性能可控性(主观) \\
    \hline
    CUDA C++ & 中等 & 很高 \\
    Triton & 较高 & 较高 \\
    纯 Python/NumPy & 很高 & 较低 \\
    \hline
  \end{tabular}
  \caption{示例表格:调研维度的占位内容}
\end{table}

\subsection{示例代码块}
下面是一个“伪 Triton 风格”的代码片段(仅用于排版示例,不保证可运行):
\begin{lstlisting}[language=Python,caption={示例代码:伪 Triton Kernel 结构}]
import triton
import triton.language as tl

@triton.jit
def add_kernel(X_ptr, Y_ptr, Z_ptr, n_elements: tl.constexpr, BLOCK: tl.constexpr):
    pid = tl.program_id(0)
    offsets = pid * BLOCK + tl.arange(0, BLOCK)
    mask = offsets < n_elements
    x = tl.load(X_ptr + offsets, mask=mask, other=0.0)
    y = tl.load(Y_ptr + offsets, mask=mask, other=0.0)
    tl.store(Z_ptr + offsets, x + y, mask=mask)
\end{lstlisting}

\section{结论与参考文献(示例)}
这份示例稿不追求内容完整性,重点是演示构建方式:
\begin{itemize}
  \item 在 \texttt{DOC\_ROOT/} 下运行 \texttt{make} 生成 \texttt{DOC\_ROOT/build/TritonSurvey.pdf}
  \item 使用 \texttt{make debug} 得到更详细的错误输出(\texttt{file:line} 形式)
  \item 使用 \texttt{make clean} / \texttt{make distclean} 管理中间产物
\end{itemize}

\clearpage
\begin{thebibliography}{9}
  \bibitem{triton}
  Triton language project.
  \newblock \url{https://github.com/triton-lang/triton}

  \bibitem{cuda_guide}
  NVIDIA.
  \newblock \emph{CUDA C++ Programming Guide}.
  \newblock \url{https://docs.nvidia.com/cuda/cuda-c-programming-guide/}
\end{thebibliography}
\par\noindent 待补充。
\clearpage


\clearpage
\phantomsection
\addcontentsline{toc}{section}{参考文献}
\bibliographystyle{plain}
\bibliography{refs}

\end{document}
