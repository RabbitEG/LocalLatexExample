\section{Triton 的核心概念(示例)}
\subsection{编程模型}
Triton 常用的描述方式是“以 block/tile 为单位的张量计算”:将大问题拆成若干 tile,每个 program 实例负责处理一块 tile,
并通过显式的内存访问模式来争取更好的带宽利用率。

\subsection{示例图片}
图 \ref{fig:placeholder} 只是为了演示图像插入与构建流程。
在 \texttt{docs/} 子工程里,我们把图片文件放在 \texttt{docs/figures/} 下,避免与仓库其它目录冲突。

\begin{figure}[H]
  \centering
  \IfFileExists{figures/placeholder.pdf}{%
    \includegraphics[width=0.72\textwidth]{figures/placeholder.pdf}%
  }{%
    \fbox{\parbox[c][4cm][c]{0.72\textwidth}{\centering Placeholder Figure}}%
  }
  \caption{示例图片(可用 placeholder.pdf,也可用占位框)}
  \label{fig:placeholder}
\end{figure}

\subsection{示例表格}
\begin{table}[H]
  \centering
  \begin{tabular}{lcc}
    \hline
    方案 & 开发体验(主观) & 性能可控性(主观) \\
    \hline
    CUDA C++ & 中等 & 很高 \\
    Triton & 较高 & 较高 \\
    纯 Python/NumPy & 很高 & 较低 \\
    \hline
  \end{tabular}
  \caption{示例表格:调研维度的占位内容}
\end{table}

\subsection{示例代码块}
下面是一个“伪 Triton 风格”的代码片段(仅用于排版示例,不保证可运行):
\begin{lstlisting}[language=Python,caption={示例代码:伪 Triton Kernel 结构}]
import triton
import triton.language as tl

@triton.jit
def add_kernel(X_ptr, Y_ptr, Z_ptr, n_elements: tl.constexpr, BLOCK: tl.constexpr):
    pid = tl.program_id(0)
    offsets = pid * BLOCK + tl.arange(0, BLOCK)
    mask = offsets < n_elements
    x = tl.load(X_ptr + offsets, mask=mask, other=0.0)
    y = tl.load(Y_ptr + offsets, mask=mask, other=0.0)
    tl.store(Z_ptr + offsets, x + y, mask=mask)
\end{lstlisting}
