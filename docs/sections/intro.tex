\section{背景与目标}
在 GPU 编程生态里,开发者常见的选择包括 CUDA C++、OpenCL,以及更高层的编程抽象(如基于 Python 的 DSL)。
Triton 的定位可以粗略理解为:在保持性能可控的同时,提供更接近 Python 科研/工程栈的开发体验 \cite{triton}.

本示例文档的目标是:
\begin{itemize}
  \item 提供一个最小但完整的可编译结构(目录、公式、表格、代码块、图片)。
  \item 演示通过 \texttt{latexmk} 自动决定“跑几遍”,并把产物集中到 \texttt{build/}。
\end{itemize}

\subsection{一个极简的性能直觉}
下面给出一个非常粗糙的估算式(仅用于示例排版):
\begin{equation}
  T \approx \frac{\text{Bytes Moved}}{\text{Memory Bandwidth}} + \frac{\text{FLOPs}}{\text{Compute Throughput}}.
\end{equation}
在工程实践中,通常会结合硬件文档和编程指南来验证这些直觉 \cite{cuda_guide}.
